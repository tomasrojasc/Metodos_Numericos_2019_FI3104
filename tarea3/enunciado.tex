\documentclass[letter, 11pt]{article}
%% ================================
%% Packages =======================
\usepackage[utf8]{inputenc}      %%
\usepackage[T1]{fontenc}         %%
\usepackage{lmodern}             %%
\usepackage[spanish]{babel}      %%
\decimalpoint                    %%
\usepackage{fullpage}            %%
\usepackage{fancyhdr}            %%
\usepackage{graphicx}            %%
\usepackage{amsmath}             %%
\usepackage{color}               %%
\usepackage{mdframed}            %%
\usepackage[colorlinks]{hyperref}%%
%% ================================
%% ================================

%% ================================
%% Page size/borders config =======
\setlength{\oddsidemargin}{0in}  %%
\setlength{\evensidemargin}{0in} %%
\setlength{\marginparwidth}{0in} %%
\setlength{\marginparsep}{0in}   %%
\setlength{\voffset}{-0.5in}     %%
\setlength{\hoffset}{0in}        %%
\setlength{\topmargin}{0in}      %%
\setlength{\headheight}{54pt}    %%
\setlength{\headsep}{1em}        %%
\setlength{\textheight}{8.5in}   %%
\setlength{\footskip}{0.5in}     %%
%% ================================
%% ================================

%% =============================================================
%% Headers setup, environments, colors, etc.
%%
%% Header ------------------------------------------------------
\fancypagestyle{firstpage}
{
  \fancyhf{}
  \lhead{\includegraphics[height=4.5em]{LogoDFI.jpg}}
  \rhead{FI3104-1 \semestre\\
         Métodos Numéricos para la Ciencia e Ingeniería\\
         Prof.: \profesor}
  \fancyfoot[C]{\thepage}
}

\pagestyle{plain}
\fancyhf{}
\fancyfoot[C]{\thepage}
%% -------------------------------------------------------------
%% Environments -------------------------------------------------
\newmdenv[
  linecolor=gray,
  fontcolor=gray,
  linewidth=0.2em,
  topline=false,
  bottomline=false,
  rightline=false,
  skipabove=\topsep
  skipbelow=\topsep,
]{ayuda}
%% -------------------------------------------------------------
%% Colors ------------------------------------------------------
\definecolor{gray}{rgb}{0.5, 0.5, 0.5}
%% -------------------------------------------------------------
%% Aliases ------------------------------------------------------
\newcommand{\scipy}{\texttt{scipy}}
%% -------------------------------------------------------------
%% =============================================================
%% =============================================================================
%% CONFIGURACION DEL DOCUMENTO =================================================
%% Llenar con la información pertinente al curso y la tarea
%%
\newcommand{\tareanro}{3}
\newcommand{\fechaentrega}{04/09/2019 23:59 hrs}
\newcommand{\semestre}{2019B}
\newcommand{\profesor}{Valentino González}
%% =============================================================================
%% =============================================================================


\begin{document}
\thispagestyle{firstpage}

\begin{center}
  {\uppercase{\LARGE \bf Tarea \tareanro}}\\
  Fecha de entrega: \fechaentrega
\end{center}


%% =============================================================================
%% ENUNCIADO ===================================================================
\noindent{\large \bf Problema 1}

El objetivo de este problema es investigar cómo se comporta la interpolación
con polinomios versus la interpolación spline en algunos casos interesantes.

Considere la siguiente función Gaussiana:

$$ f(x) = e^{-x^2/0.05}$$

\noindent en el intervalo $[-1, 1]$. Divida el intervalo en 4 tramos
equiespaciados (es decir, escoja 5 puntos en el intervalo $[-1, 1]$). Ahora
interpole un polinomio (usando, por ejemplo, el método de Lagrange) que pase
por esos 5 puntos. Haga lo mismo usando una interpolación Spline.

Ahora aumente secuencialmente el número de puntos y compruebe cómo se comportan
los dos métodos (mejoran?, empeoran?, es lo que esperaba?).

\begin{ayuda}
  \noindent {\bf Nota.}

  Puede programar su propio método de Lagrange y/o Spline, o puede utilizar
  alguna librería que le parezca adecuada. Investigue, por ejemplo, el módulo
  de interpolación de \texttt{scipy}. Si decide usar una librería, asegúrese de
  entender los detalles de las implementaciones (¿qué pasa en los extremos de
  la interpolación \texttt{spline}, por ejemplo?). Incluya esta información en
  el informe.
\end{ayuda}


\vspace{1em}
\noindent{\large \bf Problema 2}

En este problema exploraremos una de las múltiples aplicaciones para los
métodos de interpolación: la estimación de datos faltantes y la extrapolación.

El archivo \texttt{GLB.Ts+dSST-short.csv} es una archivo de datos separado por
comas. Los datos provienen del \emph{Goddard Institute for Space Science}
(GISS) y contienen información sobre las anomalías de temperatura medidas en la
tierra y los océanos a lo largo de los años. Para ser precisos, la columna
titulada \texttt{J-D} indica la diferencia entre la temperatura base (elegida
como la temperatura promedio entre los años 1951 y 1980) y el promedio
anual (\emph{January-December}) para ese año, el cual se indica en la columna
titulada \texttt{Year}.

\vspace{0.5em}
\noindent a) Estime el valor de la anomalía de temperatura promedio para el año
2016 y compárelo con el valor medido que fue 1.02. Explique cómo hizo la
estimación y por qué eligió hacerla de esa manera. ¿Puede justificar la
diferencia entre su estimación y el valor real medido?

\vspace{0.5em}
\noindent b) Estime el valor que tendrá la anomalía de temperatura promedio
para el año 2019. Explique cómo hizo la estimación y cómo esta depende del
método escogido y sus detalles.

\begin{ayuda}
  - Utilice métodos de interpolación que pasen por todos los puntos. Más
  adelante veremos otros métodos que interpolan funciones suaves sin pasar
  exactamente por todos los puntos.

  \noindent- Si le interesa obtener más información sobre los datos utilizados
  en esta pregunta, consulte la siguiente página:
  \href{https://data.giss.nasa.gov/gistemp/}{https://data.giss.nasa.gov/gistemp/}

  \noindent- Para leer el archivo \texttt{CSV} y cargarlo en un formato útil,
  le pueden ser útiles las siguientes tareas: \texttt{numpy.genfromtxt}, ó
  \texttt{pandas.read\_csv}. Revise la ayuda para aprender cómo utilizarlos.
\end{ayuda}

%% FIN ENUNCIADO ===============================================================
%% =============================================================================

\pagebreak
\noindent{\bf Instrucciones importantes.}
\begin{itemize}

  \item Utilice \texttt{git} durante el desarrollo de la tarea para mantener un
    historial de los cambios realizados. La siguiente
    \href{https://education.github.com/git-cheat-sheet-education.pdf}{cheat
      sheet} le puede ser útil. {\bf Esta vez revisaremos el uso apropiado de
    la herramienta y asignaremos una fracción del puntaje a este ítem.} Realice
    cambios pequeños y guarde su progreso (a través de \emph{commits})
    regularmente. No guarde código que no corre o compila (si lo hace por algún
    motivo deje un mensaje claro que lo indique). Escriba mensajes claros que
    permitan hacerse una idea de lo que se agregó de un \texttt{commit} al
    siguiente.

  \item También comenzaremos a revisar su uso correcto de python. Si define una
    función relativamente larga o con muchos parámetros, recuerde escribir el
    \emph{docstring} que describa los parámetros que recibe la función, el
    output, y el detalle de qué es lo que hace la función. Recuerde que
    generalmente es mejor usar varias funciones cortas (que hagan una sola cosa
    bien) que una muy larga (que lo haga todo).  Utilice nombres explicativos
    tanto para las funciones como para las variables de su código. El mejor
    nombre es aquel que permite entender qué hace la función sin tener que leer
    su implementación.

  \item La distribución de puntaje para esta tarea será: 50\% resolución
    correcta del problema (principalmente el código); 40\% calidad del informe
    (claridad del texto, calidad de las figuras o tablas si corresponde, etc);
    5\% uso apropiado de \texttt{git}; y 5\% calidad del código (en cuanto a su
    modularidad, claridad, etc.).

  \item La tarea se entrega subiendo su trabajo a github. Clone este
    repositorio (el que está en su propia cuenta privada), trabaje en el código
    y en el informe y cuando haya terminado asegúrese de hacer un último
    \texttt{commit} y luego un \texttt{push} para subir todo su trabajo a
    github.

  \item El informe debe ser entregado en formato \texttt{pdf}, este debe ser
    claro sin información de más ni de menos. Esto es importante, no escriba de
    más, esto no mejorará su nota sino que al contrario. Asegúrese de utilizar
    figuras efectivas y tablas para resumir sus resultados. Revise su
    ortografía.

\end{itemize}

\end{document}
