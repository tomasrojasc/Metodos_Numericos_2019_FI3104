\documentclass[letter, 11pt]{article}
%% ================================
%% Packages =======================
\usepackage[utf8]{inputenc}      %%
\usepackage[T1]{fontenc}         %%
\usepackage{lmodern}             %%
\usepackage[spanish]{babel}      %%
\decimalpoint                    %%
\usepackage{fullpage}            %%
\usepackage{fancyhdr}            %%
\usepackage{graphicx}            %%
\usepackage{amsmath}             %%
\usepackage{color}               %%
\usepackage{mdframed}            %%
\usepackage[colorlinks]{hyperref}%%
%% ================================
%% ================================

%% ================================
%% Page size/borders config =======
\setlength{\oddsidemargin}{0in}  %%
\setlength{\evensidemargin}{0in} %%
\setlength{\marginparwidth}{0in} %%
\setlength{\marginparsep}{0in}   %%
\setlength{\voffset}{-0.5in}     %%
\setlength{\hoffset}{0in}        %%
\setlength{\topmargin}{0in}      %%
\setlength{\headheight}{54pt}    %%
\setlength{\headsep}{1em}        %%
\setlength{\textheight}{8.5in}   %%
\setlength{\footskip}{0.5in}     %%
%% ================================
%% ================================

%% =============================================================
%% Headers setup, environments, colors, etc.
%%
%% Header ------------------------------------------------------
\fancypagestyle{firstpage}
{
  \fancyhf{}
  \lhead{\includegraphics[height=4.5em]{LogoDFI.jpg}}
  \rhead{FI3104-1 \semestre\\
         Métodos Numéricos para la Ciencia e Ingeniería\\
         Prof.: \profesor}
  \fancyfoot[C]{\thepage}
}

\pagestyle{plain}
\fancyhf{}
\fancyfoot[C]{\thepage}
%% -------------------------------------------------------------
%% Environments -------------------------------------------------
\newmdenv[
  linecolor=gray,
  fontcolor=gray,
  linewidth=0.2em,
  topline=false,
  bottomline=false,
  rightline=false,
  skipabove=\topsep
  skipbelow=\topsep,
]{ayuda}
%% -------------------------------------------------------------
%% Colors ------------------------------------------------------
\definecolor{gray}{rgb}{0.5, 0.5, 0.5}
%% -------------------------------------------------------------
%% Aliases ------------------------------------------------------
\newcommand{\scipy}{\texttt{scipy}}
%% -------------------------------------------------------------
%% =============================================================
%% =============================================================================
%% CONFIGURACION DEL DOCUMENTO =================================================
%% Llenar con la información pertinente al curso y la tarea
%%
\newcommand{\tareanro}{5}
\newcommand{\fechaentrega}{26/09/2019 23:59 hrs}
\newcommand{\semestre}{2019B}
\newcommand{\profesor}{Valentino González}
%% =============================================================================
%% =============================================================================


\begin{document}
\thispagestyle{firstpage}

\begin{center}
  {\uppercase{\LARGE \bf Tarea \tareanro}}\\
  Fecha de entrega: \fechaentrega
\end{center}


%% =============================================================================
%% ENUNCIADO ===================================================================
\noindent{\large \bf Problema 1}

\noindent{\bf El péndulo de Kapitza.} Considere una vara ideal sin masa de
largo $L=1 m$, conectada a una masa $m$ en un extremo, y en el otro extremo a un
punto que oscila sinusoidalmente en la dirección vertical con frecuencia $w_1$ y
amplitud $A=L/2$. Se puede demostrar que la ecuación de movimiento para este
péndulo es:

$$ \frac{d^2 \phi(t)}{dt^2} =
          -w_0^2 sin(\phi) - \frac{A}{L}w_1^2 cos(w_1 t) sin(\phi) $$

\noindent donde $\phi$ es el ángulo que forma la vara contra la vertical y
$w_0=\sqrt{g/L}$. Definiremos $\lambda=\frac{A w_1}{2 w_0 L}$ y estudiaremos el
comportamiento del péndulo para $\lambda=1, 2$ y $3$. Considere además las
siguientes condiciones iniciales:

\begin{enumerate}
  \item $\phi(t=0) = 0.00RRR$ y velocidad angular $0$.
  \item $\phi(t=0) = \pi - 0.00RRR$ y velocidad angular $0$.
\end{enumerate}

\noindent donde $RRR$ son los últimos 3 dígitos de su RUT, antes del dígito
verificador.

Integre la ecuación del péndulo de Kapitza utilizando su propia implementación
del algoritmo de Runge-Kutta de orden 4. Note que debe realizar la integración
para 3 valores distintos de $\lambda$ y 2 sets de condiciones iniciales, es
decir, debe hacerlo 6 veces.

Ahora elija uno de los 6 resultados anteriores --uno que le haya parecido
interesante-- y realice la integración una vez más utilizando alguna
implementación de Runge-Kutta de libre disposición (lo más sencillo es buscar
en \texttt{scipy.integrate}). Compare este nuevo resultado con el obtenido
anteriormente en términos de la precisión de la solución, y del tiempo de
ejecución.


%% FIN ENUNCIADO ===============================================================
%% =============================================================================

\vspace{1em}
\noindent{\bf Instrucciones importantes.}
\begin{itemize}

  \item Utilice \texttt{git} durante el desarrollo de la tarea para mantener un
    historial de los cambios realizados. La siguiente
    \href{https://education.github.com/git-cheat-sheet-education.pdf}{cheat
      sheet} le puede ser útil. \textbf{Esta vez revisaremos el uso apropiado
    de la herramienta y asignaremos una fracción del puntaje a este ítem.}
    Realice cambios pequeños y guarde su progreso (a través de \emph{commits})
    regularmente. No guarde código que no corre o compila (si lo hace por algún
    motivo deje un mensaje claro que lo indique). Escriba mensajes claros que
    permitan hacerse una idea de lo que se agregó y/o cambió de un
    \texttt{commit} al siguiente.

  \item También revisaremos su uso correcto de \texttt{python}. Si define una
    función relativametne larga o con muchos parámetros, recuerde escribir el
    \emph{docstring} que describa los parámetros que recibe la función, el
    output, y el detalle de qué es lo que hace la función. Recuerde que
    generalmente es mejor usar varias funciones cortas (que hagan una sola cosa
    bien) que una muy larga (que lo haga todo).  Utilice nombres explicativos
    tanto para las funciones como para las variables de su código.  El mejor
    nombre es aquel que permite entender qué hace la función sin tener que leer
    su implementación.

  \item Para \texttt{python} existe una guía de estilo sintáctico
    (\texttt{\href{https://www.python.org/dev/peps/pep-0008/}{PEP8}}) que
    entrega un set de reglas simples para crear código ordenado y fácilmente
    legible por otras personas. Por ejemplo, se recomienda no usar lineas más
    largas que 79 caracteres. Para esta tarea, su código debe aprobar
    \texttt{pep8}. En \href{http://pep8online.com}{esta página} puede chequear
    si su código aprueba \texttt{PEP8}. También hay utilidades en la línea de
    comando que permiten hacer la prueba directamente en sus computadores: 
    
    \texttt{>\ pip install pycodestyle}\\
    \texttt{>\ pycodestyle <filename>}

  \item La tarea se entrega subiendo su trabajo a github. Cuando termine
    asegúrese de hacer un último \texttt{commit} y luego un \texttt{push} para
    subir todo su trabajo a github.

  \item El informe debe ser entregado en formato \texttt{pdf}, este debe ser
    claro sin información de más ni de menos. \textbf{Esto es muy importante,
    no escriba de más, esto no mejorará su nota sino que al contrario}. Por
    ejemplo, la presente tarea probablemente no requiere informes de más de 4
    páginas en total (esto no es una regla estricta, sólo una referencia útil).
    Asegúrese de utilizar figuras efectivas y tablas para resumir sus
    resultados. Revise su ortografía.

  \item Repartición de puntaje: 45\% implementación y resolución del problema
    (independiente de la calidad de su código); 40\% calidad del reporte
    entregado: demuestra comprensión del problema y su solución, claridad del
    lenguaje, calidad de las figuras utilizadas; 5\% aprueba o no
    \texttt{PEP8} (acá es todo o nada); 5\% uso apropopiado de \texttt{git}; 5\%
    diseño del código: modularidad, uso efectivo de nombres de variables y
    funciones, docstrings, etc.

\end{itemize}

\end{document}
